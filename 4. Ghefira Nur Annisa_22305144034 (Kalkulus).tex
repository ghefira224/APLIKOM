\documentclass[a4paper,10pt]{article}
\usepackage{eumat}

\begin{document}
\begin{eulernotebook}
\begin{eulercomment}
Nama :Ghefira Nur Annisa\\
NIM  :22305144034\\
Kelas: Matematika E 2022

\begin{eulercomment}
\eulerheading{2. LIMIT FUNGSI}
\begin{eulercomment}
Materi mencakup di antaranya:\\
1. Mendefinisikan Limit Fungsi pada EMT\\
\end{eulercomment}
\begin{eulerttcomment}
   1.1 Definisi LIMIT KIRI
   1.2 Definisi LIMIT KANAN
\end{eulerttcomment}
\begin{eulercomment}
2. LIMIT FUNGSI ALJABAR\\
3. LIMIT FUNGSI NON ALJABAR (transenden)\\
\end{eulercomment}
\begin{eulerttcomment}
   3.1 Limit Fungsi Trigonometri
   3.2 Limit Fungsi Eksponensial
   4.3 Limit Fungsi Logaritma
\end{eulerttcomment}
\begin{eulercomment}


\begin{eulercomment}
\eulerheading{Definisi Limit}
\begin{eulercomment}
Dalam matematika, konsep limit digunakan untuk menjelaskan perilaku
suatu fungsi saat peubah bebasnya mendekati suatu titik tertentu, atau
menuju tak hingga; atau perilaku dari suatu barisan saat indeks
mendekati tak hingga. Limit dipakai dalam kalkulus (dan cabang lainnya
dari analisis matematika) untuk membangun pengertian kekontinuan,
turunan dan integral.\\
Dalam pelajaran matematika, limit biasanya mulai dipelajari saat
pengenalan terhadap kalkulus.

\begin{eulercomment}
\eulerheading{Limit Fungsi}
\begin{eulercomment}
Jika f(x) adalah fungsi real dan c adalah bilangan real, maka:

\end{eulercomment}
\begin{eulerformula}
\[
\lim_{x \to c} f(x) = L
\]
\end{eulerformula}
\begin{eulercomment}
Notasi tersebut menyatakan bahwa f(x) untuk niai x mendekati c sama
dengan L. F(x) disini dapat berupa bermacam-macam jenis fungsi. Dan L
dapat berupa konstanta, ataupun "und" (tak terdefinisi), "ind" (tak
tentu namun terbatas), "infinity" (kompleks tak hingga). Begitupun
dengan batas c, dapat berupa sebarang nilai atau pada tak hingga
(-inf, minf, dan inf).

Sebuah fungsi dapat dikatakan memiliki limit apabila limit kanan dan
limit kiri nya memiliki nilai yang sama. Dimana, limit dari fungsi
tersebut adalah nilai dari limit kanan dan limit kiri fungsi yang
bernilai sama tadi.
\end{eulercomment}
\begin{eulerprompt}
>                                                   
\end{eulerprompt}
\begin{eulerudf}
     
\end{eulerudf}
\begin{eulerprompt}
>     
\end{eulerprompt}
\eulerheading{Limit Kiri dan Kanan}
\begin{eulercomment}
Pengertian limit kiri dan limit kanan berkaitan dengan pendekatan
nilai sebuah fungsi saat variabel inputnya mendekati suatu nilai
tertentu dari sisi kiri atau kanan titik tersebut.


Limit kiri didefinisikan sebagai nilai yang didekati oleh fungsi saat
variabel inputnya mendekati suatu nilai tertentu dari nilai yang lebih
kecil atau dari sisi kiri.\\
Limit kiri ditulis sebagai:

\end{eulercomment}
\begin{eulerformula}
\[
\lim \limits_{x \to c^-} {f(x)} = L
\]
\end{eulerformula}
\begin{eulercomment}
Ini berarti bahwa saat x mendekati c dari sisi kiri, nilai dari fungsi
f(x) mendekati nilai L.


Sebaliknya, limit kanan didefinisikan sebagai nilai yang didekati oleh
fungsi saat variabel inputnya mendekati suatu nilai tertentu dari
nilai yang lebih besar atau dari sisi kanan.\\
Limit kanan ditulis sebagai:

\end{eulercomment}
\begin{eulerformula}
\[
\lim \limits_{x \to c^+} {f(x)} = L
\]
\end{eulerformula}
\begin{eulercomment}
Ini berarti bahwa saat x mendekati c dari sisi kanan, nilai dari
fungsi f(x) mendekati nilai L.\\
Dalam banyak kasus, untuk limit fungsi yang ada, nilai limit kiri dan
limit kanan mungkin berbeda.
\end{eulercomment}
\begin{eulerprompt}
>                        
\end{eulerprompt}
\begin{eulerudf}
   
\end{eulerudf}
\begin{eulerprompt}
>    
\end{eulerprompt}
\begin{eulerudf}
   
\end{eulerudf}
\begin{eulerprompt}
> 
\end{eulerprompt}
\eulerheading{Limit pada EMT}
\begin{eulercomment}
Pada EMT cara mendefinisikan limit yaitu dengan format :

\textdollar{}showev('limit(f(x),x,c))

Format tersebut akan menampilkan limit yang dimaksud dan hasilnya.
Jika kita ingin menampilkan hasilnya saja dari sebuah limit tanpa
menampilkan limitnya, kita bisa menggunakan format :

'limit(f(x),x,c)

Sedangkan, untuk limit kanan dan limit kiri seperti pada definisi
dapat ditampilkan di EMT dengan cara menambah opsi "plus" atau "minus"
:

\textdollar{}showev('limit(f(x),x,c, plus)) atau 'limit(f(x),x,c, minus)

Limit dapat divisualisasikan menggunakan plot 2 dimensi. Pada EMT
sendiri, format yang bisa digunakan untuk memvisualisasikan limit
adalah :

plot2d("f(x)",-c,c):

aspect(1.5); plot2d("f(x)",c); plot2d(x,c\textgreater{}points,style="ow",\textgreater{}add):

Dengan f(x) adalah fungsi pada limit yang dicari, dan c berupa
bilangan real menyesuaikan batas dari limit itu sendiri.
\end{eulercomment}
\begin{eulerprompt}
>               
\end{eulerprompt}
\eulerheading{Limit Fungsi Aljabar}
\begin{eulercomment}
Limit fungsi aljabar adalah nilai yang didekati oleh sebuah fungsi
saat variabel inputnya mendekati suatu nilai tertentu.\\
Secara matematis, kita dapat menyatakan limit fungsi aljabar sebagai
berikut:\\
Diberikan fungsi f(x), dengan x mendekati suatu nilai c, maka limit
fungsi f(x) saat x mendekati c dapat ditulis sebagai:

\end{eulercomment}
\begin{eulerformula}
\[
\lim_{x \to c} f(x) = L
\]
\end{eulerformula}
\begin{eulercomment}
Di mana L adalah nilai yang didekati oleh fungsi f(x) saat x mendekati
c. Limit fungsi ini menggambarkan perilaku fungsi pada titik c dan
dapat membantu kita mengidentifikasi apakah suatu fungsi memiliki
nilai tertentu pada suatu titik atau apakah ada asimtot vertikal atau
horizontal pada grafik fungsi tersebut.
\end{eulercomment}
\begin{eulerprompt}
>$showev('limit((x^3-13*x^2+51*x-63)/(x^3-4*x^2-3*x+18),x,3))
\end{eulerprompt}
\begin{eulerformula}
\[
\lim_{x\rightarrow 3}{\frac{x^3-13\,x^2+51\,x-63}{x^3-4\,x^2-3\,x+
 18}}=-\frac{4}{5}
\]
\end{eulerformula}
\begin{eulercomment}
penyelesaian :\\
dari persamaan polinomial diatas dapat kita masukkan limitnya yakni
x=3 kedalam persamaan tersebut, sehingga didapatkan :

\end{eulercomment}
\begin{eulerformula}
\[
\lim \limits_{x \to 3} \frac{(x^3-13x^2+51x-63)}{(x^3-4x^2-3x+18)}
\]
\end{eulerformula}
\begin{eulerformula}
\[
\lim \limits_{x \to 3} \frac{(3^3-13(3)^2+51(3)-63)}{(3^3-4(3)^2-3(3)+18)}
\]
\end{eulerformula}
\begin{eulerformula}
\[
\lim \limits_{x \to 3} \frac{(27-117+153-63)}{(27-36-9+18)}
\]
\end{eulerformula}
\begin{eulerformula}
\[
\lim \limits_{x \to 3} \frac{(0)}{0} = \infty
\]
\end{eulerformula}
\begin{eulercomment}
Maka, untuk mencari limitnya dapat dicari faktor dari persamaan
polinomial tersebut terlebih dahul, sehingga:
\end{eulercomment}
\begin{eulerprompt}
>$& factor((x^3-13*x^2+51*x-63)/(x^3-4*x^2-3*x+18))
\end{eulerprompt}
\begin{eulerformula}
\[
\frac{x-7}{x+2}
\]
\end{eulerformula}
\begin{eulercomment}
sehingga dari faktor diatas, dapat dicari limitnya yakni:\\
\end{eulercomment}
\begin{eulerformula}
\[
\lim \limits_{x \to 3} \frac{x-7}{x+2} = \lim \limits_{x \to 3} \frac{3-7}{3+2} = \frac{-4}{5}
\]
\end{eulerformula}
\begin{eulercomment}
MAKA DAOAT DIBUKTIKAN BAHWA NILAI LIMIT TERSEBUT BERNILAI BENAR
\end{eulercomment}
\begin{eulerprompt}
>aspect(1.5); plot2d("(x^3-13*x^2+51*x-63)/(x^3-4*x^2-3*x+18)",0,4); plot2d(3,-4/5,>points,style="ow",>add):
\end{eulerprompt}
\eulerimg{17}{images/4. Ghefira Nur Annisa_22305144034 (Kalkulus)-003.png}
\begin{eulerprompt}
>$showev('limit((x^2-9)/(2*x^2-5*x-3),x,3))
\end{eulerprompt}
\begin{eulerformula}
\[
\lim_{x\rightarrow 3}{\frac{x^2-9}{2\,x^2-5\,x-3}}=\frac{6}{7}
\]
\end{eulerformula}
\begin{eulerprompt}
>aspect(1.5); plot2d("(x^2-9)/(2*x^2-5*x-3)",0,4); plot2d(3,6/7,>points,style="ow",>add):
\end{eulerprompt}
\eulerimg{17}{images/4. Ghefira Nur Annisa_22305144034 (Kalkulus)-005.png}
\begin{eulerprompt}
>$showev('limit((x^2-3*x-10)/(x-5),x,5))
\end{eulerprompt}
\begin{eulerformula}
\[
\lim_{x\rightarrow 5}{\frac{x^2-3\,x-10}{x-5}}=7
\]
\end{eulerformula}
\begin{eulerprompt}
>aspect(1.5); plot2d("(x^2-3*x-10)/(x-5)",0,6); plot2d(5,7,>points,style="ow",>add):
\end{eulerprompt}
\eulerimg{17}{images/4. Ghefira Nur Annisa_22305144034 (Kalkulus)-007.png}
\begin{eulerprompt}
>$showev('limit(((2*x^2-2*x+5)/(3*x^2+x-6)),x,3))
\end{eulerprompt}
\begin{eulerformula}
\[
\lim_{x\rightarrow 3}{\frac{2\,x^2-2\,x+5}{3\,x^2+x-6}}=\frac{17}{
 24}
\]
\end{eulerformula}
\begin{eulerprompt}
>aspect(1.5); plot2d("(2*x^2-2*x+5)/(3*x^2+x-6)",0,4); plot2d(3,17/24,>points,style="ow",>add):
\end{eulerprompt}
\eulerimg{17}{images/4. Ghefira Nur Annisa_22305144034 (Kalkulus)-009.png}
\begin{eulerprompt}
>$showev('limit((3*x-6)/(x+2),x,2))
\end{eulerprompt}
\begin{eulerformula}
\[
\lim_{x\rightarrow 2}{\frac{3\,x-6}{x+2}}=0
\]
\end{eulerformula}
\begin{eulerprompt}
>aspect(1.5); plot2d("(3*x-6)/(x+2)",0,3); plot2d(2,0,>points,style="ow",>add):
\end{eulerprompt}
\eulerimg{17}{images/4. Ghefira Nur Annisa_22305144034 (Kalkulus)-011.png}
\begin{eulerprompt}
>            
\end{eulerprompt}
\begin{eulercomment}
latex:

\begin{eulercomment}
\eulerheading{Limit Fungsi Non Aljabar}
\begin{eulercomment}
\begin{eulercomment}
\eulerheading{1. Limit Fungsi Trigonometri}
\begin{eulercomment}
Limit fungsi trigonometri adalah nilai yang didekati oleh sebuah
fungsi trigonometri saat variabel inputnya mendekati suatu nilai
tertentu. Fungsi trigonometri melibatkan fungsi sinus, kosinus,
tangen, kotangen, dan sebagainya. Pada umumnya, limit fungsi
trigonometri dihitung dengan menggunakan pendekatan geometri yang
melibatkan lingkaran unit.\\
Misalnya, untuk fungsi sinus, kita dapat menyatakan limit fungsi sinus
saat x mendekati suatu nilai tertentu c sebagai:

\end{eulercomment}
\begin{eulerformula}
\[
\lim_{x \to c} sin(x) = sin(c)
\]
\end{eulerformula}
\begin{eulercomment}
Ini berarti bahwa saat x mendekati c, nilai sinus dari x akan
mendekati sinus dari c.
\end{eulercomment}
\begin{eulerprompt}
>$showev('limit(2*x*sin(x)/(1-cos(x)),x,0))
\end{eulerprompt}
\begin{eulerformula}
\[
2\,\left(\lim_{x\rightarrow 0}{\frac{x\,\sin x}{1-\cos x}}\right)=4
\]
\end{eulerformula}
\begin{eulercomment}
PENYELESAIAN :\\
\end{eulercomment}
\begin{eulerformula}
\[
2( \lim \limits_{x \to 0} \frac{xsinx}{1-cosx} )
\]
\end{eulerformula}
\begin{eulerformula}
\[
2( \lim \limits_{x \to 0} \frac{0sin(0)}{1-cos(0)} )
\]
\end{eulerformula}
\begin{eulerformula}
\[
2( \lim \limits_{x \to 0} \frac{0}{1-1} ) = \infty
\]
\end{eulerformula}
\begin{eulercomment}
MAKA kita perlu mengubah persamaan trigonometri diatas dengan
menggunakan aturan L HOSTIPAL menjadi :\\
\end{eulercomment}
\begin{eulerformula}
\[
2( \lim \limits_{x \to 0} \frac{xsin(x)}{1-cosx} )
\]
\end{eulerformula}
\begin{eulercomment}
dideferensialkan (diturunkan)\\
\end{eulercomment}
\begin{eulerformula}
\[
2( \lim \limits_{x \to 0} \frac{sinx + xcosx}{sinx} )
\]
\end{eulerformula}
\begin{eulercomment}
berdasarkan aturan L HOSPITAL persamaan trigonometri tersebut masih
menghasilkan latex: \textbackslash{}infty  \\
sehingga dapat kita turunkan lagi (dideferensialkan) menjadi:

\end{eulercomment}
\begin{eulerformula}
\[
2( \lim \limits_{x \to 0} \frac{2cosx - xsinx}{cosx} )
\]
\end{eulerformula}
\begin{eulerformula}
\[
2( \lim \limits_{x \to 0} \frac{2cos(0) - (0)sin(0)}{cos(0)} ) 
\]
\end{eulerformula}
\begin{eulerformula}
\[
2( \lim \limits_{x \to 0} \frac{2 - 0}{1} = 2 \times 2 = 4
\]
\end{eulerformula}
\begin{eulercomment}
JADI TERBUKTI BAHWA PERSAMAAN TRIGONOMETRI dari EMT TERSEBUT TERBUKTI
TERBUKTI BENAR
\end{eulercomment}
\begin{eulerprompt}
>plot2d("2*x*sin(x)/(1-cos(x))",-pi,pi); plot2d(0,4,>points,style="ow",>add):
\end{eulerprompt}
\eulerimg{17}{images/4. Ghefira Nur Annisa_22305144034 (Kalkulus)-013.png}
\begin{eulerprompt}
>$showev('limit(cot(7*h)/cot(5*h),h,0))
\end{eulerprompt}
\begin{eulerformula}
\[
\lim_{h\rightarrow 0}{\frac{\cot \left(7\,h\right)}{\cot \left(5\,h
 \right)}}=\frac{5}{7}
\]
\end{eulerformula}
\begin{eulerprompt}
>plot2d("cot(7*x)/cot(5*x)",-0.001,0.001); plot2d(0,5/7,>points,style="ow",>add):
\end{eulerprompt}
\eulerimg{17}{images/4. Ghefira Nur Annisa_22305144034 (Kalkulus)-015.png}
\begin{eulerprompt}
> $showev('limit(sin(x)/x,x,0))
\end{eulerprompt}
\begin{eulerformula}
\[
\lim_{x\rightarrow 0}{\frac{\sin x}{x}}=1
\]
\end{eulerformula}
\begin{eulerprompt}
>plot2d("sin(x)/x",-pi,pi); plot2d(0,1,>points,style="ow",>add):
\end{eulerprompt}
\eulerimg{17}{images/4. Ghefira Nur Annisa_22305144034 (Kalkulus)-017.png}
\begin{eulerprompt}
>$showev('limit(cos(2*x)/(sin(x) - cos (x)),x,0))
\end{eulerprompt}
\begin{eulerformula}
\[
\lim_{x\rightarrow 0}{\frac{\cos \left(2\,x\right)}{\sin x-\cos x}}=
 -1
\]
\end{eulerformula}
\begin{eulerprompt}
>plot2d("cos(2*x)/(sin(x) - cos (x))",-1,1):
\end{eulerprompt}
\eulerimg{17}{images/4. Ghefira Nur Annisa_22305144034 (Kalkulus)-019.png}
\begin{eulerprompt}
>$showev('limit((3*x*tan(x))/(1-cos(4*x)),x,0))
\end{eulerprompt}
\begin{eulerformula}
\[
3\,\left(\lim_{x\rightarrow 0}{\frac{x\,\tan x}{1-\cos \left(4\,x
 \right)}}\right)=\frac{3}{8}
\]
\end{eulerformula}
\begin{eulerprompt}
>plot2d("(3*x*tan(x))/(1-cos(4*x))",-pi/2,2pi,0,2pi):
\end{eulerprompt}
\eulerimg{17}{images/4. Ghefira Nur Annisa_22305144034 (Kalkulus)-021.png}
\eulerheading{2. Limit Fungsi Eksponensial}
\begin{eulercomment}
limit fungsi eksponensial adalah nilai yang didekati oleh sebuah
fungsi eksponensial saat variabel inputnya mendekati suatu nilai
tertentu. Fungsi eksponensial melibatkan bentuk fungsional seperti
a\textasciicircum{}x, dengan a sebagai basis dan x sebagai eksponen.\\
Misalnya, limit fungsi eksponensial saat x mendekati suatu nilai
tertentu c dapat dinyatakan sebagai:

\end{eulercomment}
\begin{eulerformula}
\[
\lim_{x \to c} a^x = a^c
\]
\end{eulerformula}
\begin{eulercomment}
Ini berarti bahwa saat x mendekati c, nilai dari fungsi eksponensial
a\textasciicircum{}x akan mendekati nilai a\textasciicircum{}c.\\
Limit fungsi trigonometri dan limit fungsi eksponensial memiliki
beragam sifat dan properti yang dapat digunakan dalam analisis
matematika. Mereka juga sering digunakan dalam pemodelan dan aplikasi
ilmu pengetahuan yang melibatkan perubahan atau pertumbuhan yang
berkaitan dengan sudut atau eksponensial.

\end{eulercomment}
\begin{eulerprompt}
>$showev('limit((1+2/(3*x))^(5*x),x,inf))
\end{eulerprompt}
\begin{eulerformula}
\[
\lim_{x\rightarrow \infty }{\left(\frac{2}{3\,x}+1\right)^{5\,x}}=e
 ^{\frac{10}{3}}
\]
\end{eulerformula}
\begin{eulercomment}
Penyelesaian limit fungsi eksponensial tersebut:\\
\end{eulercomment}
\begin{eulerformula}
\[
\lim \limits_{x \to \infty} \left(1+ \frac{2}{3x} \right)^{5x} = \lim \limits_{x \to \infty} \left(1+ \frac{2}{3x} \right)^{5x \cdot \frac{3}{3}}
\]
\end{eulerformula}
\begin{eulerformula}
\[
= \lim \limits_{x \to \infty} \left(1+ \frac{2}{3x} \right)^{3x \cdot \frac{5}{3}} = \left [\lim \limits_{x \to \infty} \left(1+ \frac{2}{3x} \right)^{3x} \right]^{\frac{5}{3}}
\]
\end{eulerformula}
\begin{eulerformula}
\[
= \left [\lim \limits_{3x \to \infty} \left(1+ \frac{2}{3x} \right)^{3x} \right]^{\frac{5}{3}} = \left [\lim \limits_{y \to \infty} \left(1+ \frac{2}{y} \right)^{y} \right]^{\frac{5}{3}}
\]
\end{eulerformula}
\begin{eulerformula}
\[
(e^2)^{\frac{5}{3}} = e^{\frac{10}{3}}
\]
\end{eulerformula}
\begin{eulercomment}
JADI TERBUKTI PENYELESAIAN LIMIT FUNGSI EKSPONENSIAL TERSEBUT
\end{eulercomment}
\begin{eulerprompt}
>plot2d("(1+2/(3*x))^(5*x)",-50,0,20,100):
\end{eulerprompt}
\eulerimg{17}{images/4. Ghefira Nur Annisa_22305144034 (Kalkulus)-023.png}
\begin{eulerprompt}
>$showev('limit((1+1/x)^x,x,inf))
\end{eulerprompt}
\begin{eulerformula}
\[
\lim_{x\rightarrow \infty }{\left(\frac{1}{x}+1\right)^{x}}=e
\]
\end{eulerformula}
\begin{eulerprompt}
>plot2d("(1+1/x)^x",-5,0,-1,50):
\end{eulerprompt}
\eulerimg{17}{images/4. Ghefira Nur Annisa_22305144034 (Kalkulus)-025.png}
\begin{eulerprompt}
>$showev('limit((2^(4*x)+2^(6*x))^(1/x),x,inf))
\end{eulerprompt}
\begin{eulerformula}
\[
\lim_{x\rightarrow \infty }{\left(2^{6\,x}+2^{4\,x}\right)^{\frac{1
 }{x}}}=64
\]
\end{eulerformula}
\begin{eulerprompt}
>plot2d("(2^(4*x)+2^(6*x))^(1/x)",-1,20,50,100):
\end{eulerprompt}
\eulerimg{17}{images/4. Ghefira Nur Annisa_22305144034 (Kalkulus)-027.png}
\begin{eulerprompt}
>                  
\end{eulerprompt}
\eulerheading{3. Limit Fungsi Logaritma}
\begin{eulercomment}
limit fungsi logaritma adalah nilai yang didekati oleh sebuah fungsi
logaritma saat variabel inputnya mendekati suatu nilai tertentu.
Fungsi logaritma melibatkan logaritma basis a dari x, yang ditulis
sebagai log\_a(x).\\
Misalnya, untuk fungsi logaritma alami (basis e), kita dapat
menyatakan limit fungsi logaritma saat x mendekati suatu nilai
tertentu c sebagai:

\end{eulercomment}
\begin{eulerformula}
\[
\lim_{x \to c} ln(x) = ln(c)
\]
\end{eulerformula}
\begin{eulercomment}
Ini berarti bahwa saat x mendekati c, nilai logaritma natural dari x
akan mendekati logaritma natural dari c.\\
i dan limit fungsi eksponensial memiliki beragam sifat dan properti
yang dapat digunakan dalam analisis matematika. Mereka juga sering
digunakan dalam pemodelan dan aplikasi ilmu pengetahuan yang
melibatkan perubahan atau pertumbuhan yang berkaitan dengan sudut atau
eksponensial.
\end{eulercomment}
\begin{eulerprompt}
>$showev('limit(log(x), x, minf))
\end{eulerprompt}
\begin{eulerformula}
\[
\lim_{x\rightarrow  -\infty }{\log x}={\it infinity}
\]
\end{eulerformula}
\begin{eulerprompt}
>plot2d("log(x)",0,100):
\end{eulerprompt}
\eulerimg{17}{images/4. Ghefira Nur Annisa_22305144034 (Kalkulus)-029.png}
\begin{eulerprompt}
>$showev('limit((log(2*x))^2,x,2))
\end{eulerprompt}
\begin{eulerformula}
\[
\lim_{x\rightarrow 2}{\log ^2\left(2\,x\right)}=\log ^24
\]
\end{eulerformula}
\begin{eulerprompt}
>plot2d("log(2*x)^2",0,10):
\end{eulerprompt}
\eulerimg{17}{images/4. Ghefira Nur Annisa_22305144034 (Kalkulus)-031.png}
\begin{eulerprompt}
>$showev('limit(log(2*x), x, 0))
\end{eulerprompt}
\begin{eulerformula}
\[
\lim_{x\rightarrow 0}{\log \left(2\,x\right)}={\it infinity}
\]
\end{eulerformula}
\begin{eulerprompt}
>plot2d("log(2*x)",0,100):
\end{eulerprompt}
\eulerimg{17}{images/4. Ghefira Nur Annisa_22305144034 (Kalkulus)-033.png}
\begin{eulerprompt}
>$showev('limit(log(10*x), x, 10))
\end{eulerprompt}
\begin{eulerformula}
\[
\lim_{x\rightarrow 10}{\log \left(10\,x\right)}=\log 100
\]
\end{eulerformula}
\begin{eulerprompt}
>plot2d("log(10*x)",2,10):
\end{eulerprompt}
\eulerimg{17}{images/4. Ghefira Nur Annisa_22305144034 (Kalkulus)-035.png}
\begin{eulerprompt}
>$showev('limit(log(3*x),x,0))
\end{eulerprompt}
\begin{eulerformula}
\[
\lim_{x\rightarrow 0}{\log \left(3\,x\right)}={\it infinity}
\]
\end{eulerformula}
\begin{eulerprompt}
>plot2d("log(3*x)",-10,10):
\end{eulerprompt}
\eulerimg{17}{images/4. Ghefira Nur Annisa_22305144034 (Kalkulus)-037.png}
\end{eulernotebook}
\end{document}
